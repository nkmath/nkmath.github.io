\documentclass[a4paper,onecolumn,dvipdfmx]{jsarticle}
%b4paperは用紙のサイズ,b4のところをa4に変更などすればサイズも変わる,landscapleは用紙横向き%
\usepackage[top=20truemm,bottom=20truemm,left=15truemm,right=15truemm]{geometry}
%余白の設定,単位はcm,mmでも可%


%様々なパッケージたち,パッケージ同士で衝突することもあるので注意が必要%
\usepackage{okumacro}
\usepackage{fancyhdr}
\usepackage{lastpage}
\usepackage{mathrsfs}
\usepackage[dvipdfmx,hidelinks]{hyperref}
\usepackage{pxjahyper}
\usepackage{amsmath}
\usepackage{amsfonts}
\usepackage{ascmac}
\usepackage{color}
\usepackage{amssymb}
\usepackage[dvipdfmx]{graphicx}
\usepackage[dvipdfmx]{color}
\usepackage{graphics}
\usepackage{tikz}
\usepackage{tikz-cd}
\usepackage{bm}
\usepackage{bbm}
\usepackage{picture}
\usepackage{fancybox}
\usepackage[bold]{otf}
\usepackage{stmaryrd}
\usepackage{hhline}
\usepackage{longtable}


%定理環境%
\usepackage{amsthm}

\newtheorem{dfn}{定義}
\newtheorem{thm}[dfn]{定理}
\newtheorem{prop}[dfn]{命題}
\newtheorem{lem}[dfn]{補題}
\newtheorem{cor}[dfn]{系}
\newtheorem{fact}{事実}
\newtheorem{exam}{例題}
\newtheorem{axi}{公理}
\newtheorem{prob}{問題}


%番号付けの設定,arabicがアラビア数字(1,2,3,4,...),alphはアルファベット(a,b,c,d,...),romanはローマ数字(i,ii,iii,iv,...)%
\renewcommand{\labelenumi}{\arabic{enumi}.} %一番外の番号(例えば,大問みたいな感じ)%
\renewcommand{\labelenumii}{(\alph{enumii})}%二番目の番号(例えば,大問のつぎの数字)%
\renewcommand{\labelenumiii}{\roman{enumiii}.}%三番目の番号(例えば,小問みたいな感じ)%
\renewcommand{\thefootnote}{\arabic{footnote})}%注釈の番号%


%いちいち,打つのがめんどくさかったり,覚えにくい既存のコマンドを,自分用に設定%
%以下,\newcommand{\*}{\***}タイプ:本来,"\***"と入力すべきところを"\*"と省略%
\newcommand{\s}{\mathfrak{S}}
\newcommand{\bs}{\textbackslash}
\newcommand{\txhat}{\textasciicircum}
\newcommand{\aida}[1]{\textcolor{white}{#1}}


%以下,\newcommand{\**}[n]{\***{#i}\*{#j}.../****{#k}} (n,i,j,kは何かしらの数値;i,j,kはn以下の数値で相異なる)タイプ:nは引数,"\**{#1}{#2}...{#n}"と入力すれば,"\***{#i}\*{#j}.../****{#k}"が実行される%
\newcommand{\lfrac}[2]{\left(\frac{#1}{#2}\right)}

%既存の環境を自分でオリジナルな環境に設定,最近使い始めたのでちゃんとは理解していない%
\newenvironment{smallpmatrix}{\left(\begin{smallmatrix}}{\end{smallmatrix}\right)}
\newenvironment{smallvmatrix}{\left|\begin{smallmatrix}}{\end{smallmatrix}\right|}
\newenvironment{smallbmatrix}{\left[\begin{smallmatrix}}{\end{smallmatrix}\right]}

\renewcommand{\footrulewidth}{0.4pt}%下の線%
\setlength{\columnseprule}{0.4pt}%真ん中の線%
\pagestyle{empty}
%\lhead{数学}%枠外右上の文字%
%\lfoot{日付}%枠外左下の文字%
%\rfoot{担当:〇〇}%枠外右下の文字%
%\rhead{\underline{\qquad\qquad}\,年\ \underline{\qquad\qquad}\,組\ \underline{\qquad\qquad}\,番\quad 氏名\,\underline{\hspace{20zw}}}%枠外左上の文字%

\hypersetup{ pdftitle={worksheet_sample_yoko},pdfauthor={Kentaro NAKAHASHI}}


\begin{document}
	\begin{center}
		\begin{tabular}{|p{6zw}|p{14zw}|p{14zw}|p{7.5zw}|p{7zw}|}
			\hline
			
			\hline\multicolumn{5}{|c|}{本時案}\\\hhline{|=====|}
			学習内容&学習活動&指導過程&指導上の留意点&観点・評価規準\\\hline
			導入&1.本時の目標をかく.&1.本時の目標を伝える.& & \\
			($5\sim8$分)&&&&\\
			&\multicolumn{2}{c|}{\fbox{めあて:確率を実生活に\,\textbf{応用}\,できるようになろう!}}& & \\
			&&&& \\\hline
			展開&&&&\\
			(20分)&2.問題を読み取る.&2.問題を板書する.&&\\
			      &{\footnotesize\vspace{-1.5zw}\begin{screen}
			      		問. 10000人に1人の割合で人間に感染しているウイルスがある.\,\,\,このウイルスに「感染している」「感染していない」を調べる検査の精度は$99\%$である.\,\,\,ある少年はこの検査を受け,「感染している」という結果が出た.\ \ このとき,少年が\ \underline{\textbf{``実際に"}感染している確率}\ は約何$\%$であるか?
			      \end{screen}}&\aida{2.}(プリントを配布する.)&{\footnotesize *生徒が問題を理解できてない様子の時は,補足する.}&\MARU{1}{ \footnotesize 数学への関心・意欲・態度 }-[\aida{i}i\aida{i}],
		      
		      \MARU{2}{ \footnotesize 数学的な見方・考え方 }-[\aida{i}i\aida{i}]\\
		    &3.問題を解こうとする.&3.状況に応じてヒントを出す.&&\MARU{1}{ \footnotesize 数学への関心・意欲・態度 }-[\aida{i}i\aida{i}],
		    
		    \MARU{2}{ \footnotesize 数学的な見方・考え方 }-[\aida{i}i\aida{i}],\ \MARU{3}{ \footnotesize 数学的な技能}[\aida{i}i\aida{i}],\ \MARU{4}{ \footnotesize 数学についての知識・理解 }-[\aida{i}i\aida{i}]-[\,ii\,]\\\
		    &&&&\\
		    &4.グループで話し合う.&4.グループ活動させる.
		    
		    \aida{4.}机間指導する.&&\MARU{1}{ \footnotesize 数学への関心・意欲・態度 }-[\,ii\,],\ \MARU{4}{ \footnotesize 数学についての知識・理解 }-[iii]\\
		    &&&&\\
		    &&&&\\
		    &&&&\\
		    (12分)&5.代表で2グループほどが発表
		    
		    \aida{5.}する.\ \ また,それを聞く.&5.2つのグループをこちらで選
		    
		    \aida{5.}んで発表させる.&{\footnotesize **違う意見になるように選ぶ.}&\MARU{4}{ \footnotesize 数学についての知識・理解 }-[iii]\\
		    
		    &&&&\\
		    
		    &&&&\\
		    
		    &&&&\\
		    
		    &&&&\\\hline
		    
		    まとめ
		    
		    (10分)&6.話を聞く.&6.発表の講評を行う.&{\footnotesize ***生徒に自信を失わせないよう細心の注意を払う.}&\\
		    
		    &7.話を聞く.\ \ 板書をプリントに
		    
		    \aida{7.}記入する.&7.本時のまとめを行う.
		    
		    \aida{7.}次回は問題演習を行う.& &\\\hline\hline
		    
		    \hline\multicolumn{1}{|l}{\textbf{備考}}&\multicolumn{1}{|l}{使用教科書$:$}&\multicolumn{1}{|l}{準備物$:$}&&\\
		    
		    \multicolumn{1}{|l}{}&\multicolumn{1}{|l}{改訂版 数学A 数研出版}&\multicolumn{1}{|l}{ワークシート}&&\\\hline
		    
		\end{tabular}
	\end{center}	
	
	
	
	
	
	
	
	
	
	
	
	
	
	
	
	
	
	
	
	
	
	
\end{document}