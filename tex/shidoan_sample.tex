\documentclass[a4paper,onecolumn,dvipdfmx]{jsarticle}
%b4paperは用紙のサイズ,b4のところをa4に変更などすればサイズも変わる,landscapleは用紙横向き%
\usepackage[top=10truemm,bottom=15truemm,left=15truemm,right=15truemm]{geometry}
%余白の設定,単位はcm,mmでも可%



%様々なパッケージたち,パッケージ同士で衝突することもあるので注意が必要%
\usepackage{okumacro}
\usepackage{fancyhdr}
\usepackage{lastpage}
\usepackage{mathrsfs}
\usepackage[dvipdfmx,hidelinks]{hyperref}
\usepackage{pxjahyper}
\usepackage{amsmath}
\usepackage{amsfonts}
\usepackage{ascmac}
\usepackage{color}
\usepackage{amssymb}
\usepackage[dvipdfmx]{graphicx}
\usepackage[dvipdfmx]{color}
\usepackage{graphics}
\usepackage{tikz}
\usepackage{tikz-cd}
\usepackage{bm}
\usepackage{bbm}
\usepackage{picture}
\usepackage{fancybox}
\usepackage[bold]{otf}
\usepackage{stmaryrd}
\usepackage{hhline}
\usepackage{longtable}
%\usepackage{ceo}


%定理環境%
\usepackage{amsthm}
\usepackage[xcolor]{mdframed}
\usepackage{silence}\WarningFilter{mdframed}{You got a bad break}
\let\oldtheorem=\newtheorem
\mdfdefinestyle{leftbar}{topline=false,rightline=false,bottomline=false,innertopmargin=0pt,linewidth=1pt,linecolor=black}
\newcommand{\NewTheorem}[2][linecolor=black]{\surroundwithmdframed[style=leftbar,#1]{#2}\oldtheorem{#2}}
\newcommand{\NewTheoremStar}[2][linecolor=black]{\surroundwithmdframed[style=leftbar,#1]{#2}\oldtheorem*{#2}}
\makeatletter\renewcommand{\newtheorem}{\@ifstar{\NewTheoremStar}{\NewTheorem}}\makeatother
\newtheoremstyle{mydefinition}{\topsep}{\topsep}{\rmfamily}{0pt}{\sffamily\gtfamily\bfseries}{.}{.5em}
{\thmname{#1}\thmnumber{ #2}\thmnote{ (#3)}\rmfamily}
\theoremstyle{mydefinition}


\newtheorem[topline=true,rightline=true,bottomline=true,linecolor=cyan,linewidth=2pt,shadow=true,shadowcolor=cyan!70!white,shadowsize=5pt]{dfn}{定義}
\newtheorem[topline=true,rightline=true,bottomline=true,linecolor=magenta,linewidth=2pt,shadow=true,shadowcolor=magenta!70!white,shadowsize=5pt]{thm}[dfn]{定理}
\newtheorem[topline=true,rightline=true,bottomline=true]{prop}[dfn]{命題}
\newtheorem[topline=true,rightline=true,bottomline=true]{lem}[dfn]{補題}
\newtheorem[topline=true,rightline=true,bottomline=true]{cor}{系}
\newtheorem[topline=true,rightline=true,bottomline=true,linewidth=2pt]{fact}{事実}
\newtheorem[topline=true,rightline=true,bottomline=true]{Exam}{応用例題}
\newtheorem[bottomline=true]{ex}{例}
\newtheorem{axi}{公理}
\newtheorem[rightline=true,linewidth=2pt]{prob}{問題}
\newtheorem[topline=true,rightline=true,bottomline=true]{prc}{練習}



%番号付けの設定,arabicがアラビア数字(1,2,3,4,...),alphはアルファベット(a,b,c,d,...),romanはローマ数字(i,ii,iii,iv,...)%
\renewcommand{\labelenumi}{\arabic{enumi}.} %一番外の番号(例えば,大問みたいな感じ)%
\renewcommand{\labelenumii}{(\arabic{enumii})}%二番目の番号(例えば,大問のつぎの数字)%
\renewcommand{\labelenumiii}{\roman{enumiii}.}%三番目の番号(例えば,小問みたいな感じ)%
\renewcommand{\thefootnote}{\arabic{footnote})}%注釈の番号%


%いちいち,打つのがめんどくさかったり,覚えにくい既存のコマンドを,自分用に設定%
%以下,\newcommand{\*}{\***}タイプ:本来,"\***"と入力すべきところを"\*"と省略%
\newcommand{\s}{\mathfrak{S}}
\newcommand{\bs}{\textbackslash}
\newcommand{\txhat}{\textasciicircum}
\newcommand{\aida}[1]{\textcolor{white}{#1}}


%以下,\newcommand{\**}[n]{\***{#i}\*{#j}.../****{#k}} (n,i,j,kは何かしらの数値;i,j,kはn以下の数値で相異なる)タイプ:nは引数,"\**{#1}{#2}...{#n}"と入力すれば,"\***{#i}\*{#j}.../****{#k}"が実行される%
\newcommand{\lfrac}[2]{\left(\frac{#1}{#2}\right)}

%既存の環境を自分でオリジナルな環境に設定,最近使い始めたのでちゃんとは理解していない%
\newenvironment{smallpmatrix}{\left(\begin{smallmatrix}}{\end{smallmatrix}\right)}
\newenvironment{smallvmatrix}{\left|\begin{smallmatrix}}{\end{smallmatrix}\right|}
\newenvironment{smallbmatrix}{\left[\begin{smallmatrix}}{\end{smallmatrix}\right]}

%\renewcommand{\footrulewidth}{0.4pt}%下の線%
%\setlength{\columnseprule}{0.4pt}%真ん中の線%
\pagestyle{empty}%fancy
%\lhead{数学}%枠外右上の文字%
%\lfoot{日付}%枠外左下の文字%
%\rfoot{担当:〇〇}%枠外右下の文字%
%\rhead{\underline{\qquad\qquad}\,年\ \underline{\qquad\qquad}\,組\ \underline{\qquad\qquad}\,番\quad 氏名\,\underline{\hspace{20zw}}}%枠外左上の文字%

\hypersetup{ pdftitle={教育実習_研究授業_20200925},pdfauthor={Kentaro NAKAHASHI}}


\begin{document}
	\begin{center}
		{\large 数学科(数学I\!I)学習指導案}
	\end{center}
	
	\begin{flushright}
		指導者:教育実習生 中橋 健太郎
	\end{flushright}
	
	
	\begin{enumerate}
		\item 日\textcolor{white}{ああ}時 \qquad 令和◯年◯月◯日(金) 第◯校時(00:00$\sim$00:00) ◯◯教室\\
		
		\item 対象学級\qquad 2年◯組 理系クラス($m$名:男子$k$名,女子$\ell$名)\\
		
		\item 使用教材\qquad 教科書:改訂版 高等学校 数学I\!I(数研出版)
		
		\textcolor{white}{使用教材\qquad}副教材:4プロセス 数学I\!I$+$\,B(数研出版),ワークシート
		
		\textcolor{white}{使用教材\qquad}使用機材:モニター\\
		
		\item 生\hspace{.5zw}徒\hspace{.5zw}観
		
		\quad 対象の生徒は...\\
		
		\item 単\hspace{.5zw}元\hspace{.5zw}名\qquad 第5章 指数関数と対数関数\qquad 第2節 対数関数\\
		
		\item 単元の目標
		
		\begin{enumerate}
			\item 対数の定義や性質を理解し活用することができる.
			
			\item 対数関数の式とグラフとの関係について多面的に考察し,それらの特徴を理解することができる.
			
			\item 対数の考え方を応用し様々な問題を解くことができる.\\
		\end{enumerate}
		
		
		\item 単元の指導計画
		
		\begin{tabular}{llll}
			$\bullet$\ \ 対数とその性質&(3時間)\\
			$\bullet$\ \ 対数関数&(5時間)\\
			$\bullet$\ \ 常用対数&(2時間;本時はこの2時間目)&&\\
		\end{tabular}\\
		\ \\
		\item 単元の評価規準\\
		
		\begin{tabular}{|p{11.5zw}|p{11.5zw}|p{11.5zw}|p{11.5zw}|l}
			\hhline{----}
			\centering
			\begin{tabular}{c}
				【A】\\
				関心・意欲・態度
			\end{tabular}
		    &
		    \centering
		    \begin{tabular}{c}
		    	【B】\\
		    	数学的な見方・考え方
		    \end{tabular}
	        &
	        \centering
	        \begin{tabular}{c}
	        	【C】\\
	        	数学的な技能
	        \end{tabular}
	        &
	        \centering
	        \begin{tabular}{c}
	        	【D】\\
	        	知識・理解
	        \end{tabular}
            &\\\hhline{|-|-|-|-|}
	        $\MARU{1}$対数の定義や性質を理解しようとする.\ \ $\MARU{2}$対数関数とそのグラフや値の変化に興味をもつ.\ \ $\MARU{3}$対数の性質を用いて方程式・不等式の解や,関数の最大値や最小値を求めようとする.
	        &$\MARU{1}$対数関数のグラフから値の大小をとらえることができる.\ \ また,その思考の過程を振り返り,方程式や不等式を多面的に考察することができる.
	        &$\MARU{1}$対数関数の考え方において事象を数学的に表現・処理する仕方や推論の方法などの技能を身につけている.\ \ $\MARU{2}$対数関数について理解した事柄を他者に説明できる.
	        &$\MARU{1}$対数の定義・性質が理解できる.\ \ $\MARU{2}$対数関数の意味や対数の性質を理解し,基礎的な知識を身につけている.\ \ $\MARU{3}$ 対数関数と指数関数の関係性について理解できる.\\\hhline{----}
		\end{tabular}
		
		\ \\
		
		\item 本時の目標
		
		片対数グラフについて理解し,その考え方を現実問題に応用できる.
		\begin{itemize}
			\item 片対数グラフと指数関数のグラフの関係性が理解できる.\hfill 【D】
			\item 片対数グラフをかくことができる.\hfill 【C】
			\item 常用対数表を利用して$\alpha^\beta$の値を求めることができる.\hfill 【C】
		\end{itemize}
		
		
		
		\newpage
		\item 本時の展開
		
		
			\begin{longtable}{|p{4zw}|p{19zw}|p{16zw}|p{6zw}|}
				\hline
				
				\hline
				時間&学習内容および学習活動&指導上の留意点&評価(観点)\\\hline
				導入
				
				(5分)&$\bullet$\ 前回までの復習を行う.&$\bullet$\ 常用対数表について復習する.&\\\hline
				
				展開
				
				(40分)
				
				[5分]&
				$\bullet$\ 世界の人口推移について考える.\vspace{1.5zw}
				
				$\bullet$\ 人口推移の表から2015年の人口がどれくらいであるか予測する.\vspace{0.5zw}
				
				$\bullet$\ 予測を発表する.
			    &
			    $\bullet$\ プリントを配布する.
			    &{\footnotesize ・積極的に取り組んでいるか【A】}
			    \\
			    &&&\\
			    &&&\\
			    ~\vspace{9.75zw}
			    
			    [5分]&~\vspace{9.75zw}
			    
			    $\bullet$\ 片対数グラフをかく.\vspace{3.25zw}
			    
			    $\bullet$\ グラフから2015年の人口を予測する.
			    
			    &$\bullet$\ 片対数グラフについて説明する.\ \ 特に指数関数を片対数グラフでかくと直線となることを確認させ,2点$(a_1,b_1),(a_2,b_2)$を通る直線の傾きは$(\log_{10}b_2-\log_{10}b_1)/(a_2-a_1)$であることを強調する.\vspace{1zw}
			    
			    $\bullet$\ グラフの概形が直線状になっていることを確認させる.\vspace{3zw}
			    
			    $\bullet$\ プリントの空欄を埋めさせる.
			    &
			    {\footnotesize ~\vspace{12.75zw}
			    	
			    	・積極的に取り組んでいるか【A】
			    	
			    	・片対数グラフをかくことができる【C】
			    }\\
		        &&&\\
		        &&&\\
		        
		        [10分]&$\bullet$\ 片対数グラフの考え方を用いて計算し,2015年の人口を予測する.\vspace{0.5zw}
		        
		        $\bullet$\ 片対数グラフにおける直線の方程式を求める.
		        
		        &$\bullet$\ 片対数グラフの考え方を用いて計算し,2015年の人口を予測させる.\vspace{2.5zw}
		        
		        $*$\ 計算の際,常用対数表を用いることを強調する.
		        
		        
		        &
		        {\footnotesize ・片対数グラフの直線の方程式を求めることができる【C】
		        	
		        	・$10^{\alpha}$の値を常用対数表から求めることができる【C】}\\
	        	&&&\\
	        	&~\vspace{12zw}
	        	
	        	$\bullet$\ 片対数グラフが活用されている例として,最近ではcovid-19があることを知る.
	        	&$\bullet$\ 前回の常用対数表の使い方と逆であることを強調する.\vspace{0.5zw}
	        	
	        	$\bullet$\ 常用対数と桁数の関係性に注意しながら,解説を行う.\vspace{1zw}
	        	
	        	$\bullet$\ 実際の結果と誤差について言及する.&\\
	        	&&&\\
	        	&&&\\\hline
	        	まとめ
	        	
	        	(5分)
	        	&
	        	$\bullet$\ 本日のまとめ.&$\bullet$\ 片対数グラフを利用することで先を見通しやすくなることと常用対数表の使い方について確認する.&\\\hline
		    
				
			\end{longtable}
		
		
		
		
		
	\end{enumerate}
	
	
	
		
	
	
	
	
	
	
	
	
	
	
	
	
	
	
	
	
	
	
	
	
	
	
\end{document}